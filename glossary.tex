\makeglossaries

% Acronyms
% Follow the form: \newacronym{label}{ACRONYM}{expansion}
% Also: alphabetize!
\newacronym{aws}{AWS}{Amazon Web Services}
\newacronym{caida}{CAIDA}{Center for Applied Internet Data Analysis}
\newacronym{cv}{CV}{coefficient of variance}
\newacronym{ddos}{DDoS}{Distributed Denial of Service}
\newacronym{dns}{DNS}{domain name system}
\newacronym{fcc}{FCC}{Federal Communications Commission}
\newacronym{gis}{GIS}{geographic information system}
\newacronym{gps}{GPS}{global positioning system}
\newacronym{http}{HTTP}{Hypertext Transfer Protocol}
\newacronym{ip}{IP}{internet protocol}
\newacronym{isp}{ISP}{internet service provider}
\newacronym{json}{JSON}{javascript object notation}
\newacronym{mbps}{Mbps}{megabits per second}
\newacronym{ntp}{NTP}{Network Time Protocol}
\newacronym{ripe}{RIPE}{R\'eseaux IP Europ\'eens}
\newacronym{rtt}{RTT}{round trip time}
\newacronym{tld}{TLD}{top level domain}
\newacronym{ttl}{TTL}{time-to-live}
\newacronym{acUrl}{URL}{Uniform Resource Locator}
\newacronym{voip}{VoIP}{voice over internet protocol}

% Acronyms with definitions
% Follow the form: \newacronym{label}{ACRONYM}{expansion}{description}. Description appears in glossary only.
 \newacronym{cdn}{CDN}{content delivery network}{A Content Distribution Network (CDN), sometimes called a Content Delivery Network, is a network of proxy servers that form a kind of cache used to enhance delivery of content to internet users. Although helpful for internet users, they complicate measurements of connectivity to websites \textit{actually} connecting to the site's servers}
 
 \newacronym{icmp}{ICMP}{internet control message protocol}{Internet Control Message Protocol (ICMP) is a protocol designed for error reporting and other utility purposes across the internet, typically used most by routers and other intermediary devices}
 
%  \newacronym{cv}{CV}{coefficient of variance}{Coefficients of variance (CVs) are calculated by dividing the standard deviation of a data set by its mean. CVs are dimensionless values that can be judged independent of the data set, making them useful for gauging the spread of any data set. The lower the CV, the lower the spread of the data and the better the quality}
 
 \newacronym{etl}{ETL}{extract, transform, load}{ETL is a generic procedure for extracting data, transforming it into a more useful format, and loading it into a large volume storage system, such as a database. The term closer describes an architecture rather than a specific algorithm}
 

% Glossary entries -- these should be more detailed.
% Follow the form: \newglossaryentry{LABEL}{name={NAME} description={YOUR TEXT HERE}}
\newglossaryentry{favicon}{
    name={favicon},
    description={Favicons are small identifying images, typically logos or relevant UI elements, that most websites provide for browsers to place on tabs and bookmarks. Favicons are stored as .ico files (icons) and are either 16x16, 32x32, or 48x48. This small size makes them ideal for testing connection \rtt from a browser}
}

\newglossaryentry{backbone}{
    name={internet backbone},
    description={Internet backbone infrastructure consists of very high speed principal data routes and the connected major computer networks and core routers. For any trip of serious distance (likely most connections, unless you happen to have a data center in your backyard), packets will inevitably pass through some element of the internet backbone}
}

\newglossaryentry{traceroute}{
    name={traceroute},
    description={A traceroute is a technique that shows the full path data takes to get from your computer to a remote server, and how long it takes to get to each server along the way. Traceroutes leverage \icmp and specifically-set \ttl values to inciteintermediate servers to respond with an \icmp packet indicating the data packet's \ttl has expired -- thereby giving away the server's presence along the route}
}
