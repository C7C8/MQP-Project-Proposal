\makeglossaries

% Acronyms
% Follow the form: \newacronym{label}{ACRONYM}{expansion}
% Also: alphabetize!
\newacronym{api}{API}{application programming interface}
\newacronym{aws}{AWS}{Amazon Web Services}
\newacronym{caida}{CAIDA}{Center for Applied Internet Data Analysis}
\newacronym{cdf}{CDF}{cumulative distribution function}
\newacronym{cli}{CLI}{command-line interface}
\newacronym{cors}{CORS}{cross-origin resource sharing}
\newacronym{csv}{CSV}{comma separated value}
\newacronym{ddos}{DDoS}{distributed denial of service}
\newacronym{dns}{DNS}{domain name system}
\newacronym{dsl}{DSL}{digital subscriber line}
\newacronym{fcc}{FCC}{Federal Communications Commission}
\newacronym{gdp}{GDP}{gross domestic product}
\newacronym{gis}{GIS}{geographic information system}
\newacronym{gps}{GPS}{global positioning system}
\newacronym{html}{HTML}{Hypertext Markup Language}
\newacronym{http}{HTTP}{hypertext transfer protocol}
\newacronym{httpse}{HTTPS}{hypertext transfer protocol secure}
\newacronym{idw}{IDW}{inverse distance weighting}
\newacronym{ip}{IP}{internet protocol}
\newacronym{ipvs}{IPv6}{internet protocol version 6}
\newacronym{ipvf}{IPv4}{internet protocol version 4}
\newacronym{json}{JSON}{Javascript Object Notation}
\newacronym{mbps}{Mbps}{megabits per second}
\newacronym{mqp}{MQP}{major qualifying project}
\newacronym{ntp}{NTP}{Network Time Protocol}
\newacronym{ripe}{RIPE}{R\'eseaux IP Europ\'eens}
\newacronym{rtt}{RTT}{round trip time}
\newacronym{sdk}{SDK}{software development kit}
\newacronym{svg}{SVG}{scalable vector graphics}
\newacronym{tcp}{TCP}{transmission control protocol}
\newacronym{tld}{TLD}{top level domain}
\newacronym{tls}{TLS}{transport layer security}
\newacronym{ttl}{TTL}{time-to-live}
\newacronym{acUrl}{URL}{uniform resource locator}
\newacronym{us}{US}{United States}
\newacronym{voip}{VoIP}{voice over internet protocol}
\newacronym{wpi}{WPI}{Worcester Polytechnic Institute}

% Acronyms with definitions
% Follow the form: \newacronym{label}{ACRONYM}{expansion}{description}. Description appears in glossary only.
\newacronym{anova}{ANOVA}{analysis of variance}{Analysis of variance is a class of statistical models and methods for estimation, used to analyze the difference between the means of a sample. Although there are many types, they all fundamentally calculate the probability (a $p$ value) that two population means are equal.}

\newacronym{cdn}{CDN}{content delivery network}{A content distribution network, sometimes called a Content Delivery Network, is a network of proxy servers that form a kind of cache used to enhance delivery of content to internet users. Although helpful for internet users, they complicate measurements of connectivity to websites \textit{actually} connecting to the site's servers}
 
\newacronym{cv}{CV}{coefficient of variation}{Coefficients of variation, or relative standard deviations, are defined as the ratio of the absolute value of the mean of a variable divided by its standard deviation: $\frac{\lvert\mu\rvert}{\sigma}$. CVs are dimensionless values that can be judged independent of the original source, making them useful for gauging the spread of any data set. The lower the CV, the lower the spread of the data and the better the quality.}
 
\newacronym{ecc}{EC2}{Amazon Elastic Compute Cloud}{EC2 is a service from Amazon Web Services that provides virtual machines in the cloud for general-purpose or task optimized work. EC2 can be configured for different performance and pricing classes, as well as complex auto-scaling schemes or virtual private cloud setups.}
 
\newacronym{etl}{ETL}{extract-transform-load}{Extract-transform-load is a generic procedure for extracting data, transforming it into a more useful format, and loading it into a large volume storage system, such as a database. The term closer describes an architecture rather than a specific algorithm.}
 
\newacronym{icmp}{ICMP}{internet control message protocol}{Internet Control Message Protocol is a protocol designed for error reporting and other utility purposes across the internet, typically used most by routers and other intermediary devices.}

\newacronym{isp}{ISP}{internet service provider}{An internet service provider is typically the "last mile" organization that provides a user with internet access --- otherwise known as a \textit{tier 3} ISP. They are distinct from tier 2 and tier 1 ISPs which are responsible for much of the internet backbone, although ISP corporations may operate on multiple tiers. Common ISPs in the US include AT\&T, Comcast, and Verizon.}
 
\newacronym{kde}{KDE}{kernel density estimation}{Kernel density estimation is a technique for estimating the probability density function of a variable. Briefly, KDEs work by processing each measurement of a variable as if it was at the center of some given probability density function, e.g. a gaussian curve. These curves are then summed together to form one curve and normalized so the area underneath the curve is equal to 1. A KDE chart can be read in the same way as a histogram can, but the $y$ axis corresponds to a density instead of an absolute value. The advantage to using a KDE over a histogram is that KDEs are not vulnerable to binning effects (from choosing the wrong bin size) while histograms are.}

% Glossary entries -- these should be more detailed.
% Follow the form: \newglossaryentry{LABEL}{name={NAME} description={YOUR TEXT HERE}}
% \newglossaryentry{favicon}{
%     name={favicon},
%     description={Favicons are small identifying images, typically logos or relevant UI elements, that most websites provide for browsers to place on tabs and bookmarks. Favicons are stored as .ico files (icons) and are either 16x16, 32x32, or 48x48. This small size makes them ideal for testing connection \rtt from a browser}
% }

% \newglossaryentry{backbone}{
%     name={internet backbone},
%     description={Internet backbone infrastructure consists of very high speed principal data routes and the connected major computer networks and core routers. For any trip of serious distance (likely most connections, unless you happen to have a data center in your backyard), packets will inevitably pass through some element of the internet backbone}
% }

% \newglossaryentry{traceroute}{
%     name={traceroute},
%     description={A traceroute is a technique that shows the full path data takes to get from your computer to a remote server, and how long it takes to get to each server along the way. Traceroutes leverage \icmp and specifically-set \ttl values to inciteintermediate servers to respond with an \icmp packet indicating the data packet's \ttl has expired -- thereby giving away the server's presence along the route}
% }
