\section{Conclusion}

As shown in \cref{fig:caida_confidence_intervals}, the confidence intervals for each state's mean are very large, indicating that there are very few meaningful comparisons to be made here. The same conclusion can be drawn from the indistinguishability graph in \cref{fig:caida_network_invalid_comps}; there are no disjoint subgraphs so a ranking by groups of states can't even be drawn. The topological sorting results shown in \cref{tab:caida_topological_state_rankings} are interesting but they have the fundamental flaw of topological sorts as described in \cref{sec:methods_stats_topological_rankings}: the sorting relies on implicit comparisons that cannot be made according to this data set.

The underyling problem isn't with variance in the data itself (as shown by \cref{fig:caida_cv_distribution} and \cref{fig:caida_stdev_distribution} the data spread is low, proving quality data), but in the aggregation method. As visualized in \cref{fig:caida_idw_heatmap} there is simply too much variation within a state, prohibiting a statistically-valid ranking all 50 of them. We can make simple assertions where confidence intervals don't overlap, or between two states where $p<0.05$, but unfortunately nothing beyond that.

The conclusions that we can draw on a state level are that, loosely speaking, states that are more populated or more urban tend to have better internet than those are are less populated or more rural on average. The District of Columbia, being only a city and also the seat of the \us federal government, naturally has the best internet. On the other hand, rural or more sparsely populated states like Montana or North Dakota rank at absolute last. To get an accurate idea of internet connectivty, the only statistically-valid choice is to use aggregation by a narrower area, or a continuous interpolation method like the \idw heatmap in \cref{fig:caida_idw_heatmap}.
