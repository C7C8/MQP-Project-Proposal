\section{Domain Name System}\label{sec:background_dns}
\dns, a key component of internet infrastructure and connectivity, is a hierarchical system responsible for converting domain names to IP addresses. Domain names are human-readable names used to identify servers and websites. They are easier to remember than IP addresses, and can point to more than one IP address depending on geographic location or load-balancing constraints to provide higher-performance access to websites for end-users. Domain names contain one or more parts called labels, typically separated by dots. The rightmost label is the top-level domain (e.g. `com`, `org`, `net`, etc.). The DNS hierarchy tree subdivides into "zones," with each zone containing one or more domain names and sub-domains. Databases of domain names are maintained by DNS Name Servers. There are two types of name servers: Authoritative and Recursive.

\subsection{Authoritative Servers}
Authoritative servers are the primary source for the domain names within a given domain. When querying for any domain, the answer will ultimately come from the authoritative server for that domain. a \dns client can query an authoritative server directly, but it is more common that an authoritative server will be queried by a recursive server on behalf of a \dns client.

\subsection{Recursive Servers}
Recursive \dns servers work to "hunt down an IP address" for a client, so that the client does not have to do the leg work of searching through other \dns servers \cite{CloudflareWhatDNS}. Certain \dns resolvers, like those built into the user's operating system, will look at a predetermined upstream recursive \dns server, such as one provided by an \isp \cite{OracleRecursiveWork}. This server then checks its cache and, if it has a valid answer, returns it. Otherwise, the server makes a series of iterative queries in order to find the requested \ip address. 

Take for example, the \acUrl \texttt{www.google.com}. If the recursive server has no knowledge of any parts of the \acUrl, it will first query a pre-configured root \dns server for the authoritative server for the \texttt{.com} \tld. It will then query that server for \texttt{google.com}, and then finally, query the server provided from that request, Google's authoritative \dns server, for \texttt{www.google.com}. At this point, the iterative component of the lookup is complete. At each stage of this process, the recursive server caches the results of its query, and assuming the \ttl has not expired, will use that cached value instead of making a fresh query. Finally, the recursive server then returns the \ip address it located, completing the recursive request from the user.

\textcolor{red}{Diagram here or above} % Probably put a diagram somewhere in here?

\subsubsection{Public Recursive \dns Servers}
An important part of this project involves public recursive \dns servers. These servers are recursive \dns servers configured to respond to recursive requests from the public at large. Whereas most \isp recursive servers only respond to requests from paying customers, public recursive servers have no such requirement. Some organizations, like Google and CloudFlare, provide public \dns provide such services because they believe doing so improve the browsing experiences for end users \cite{GoogleIntroductionDNS}. These provide the public with \dns options outside of their \isp and provide this project with an important set of geographically diverse servers.