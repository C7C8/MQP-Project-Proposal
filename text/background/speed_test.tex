\section{Speed Test}\label{sec:speed_test_background}

Speed tests are common measures used by consumers to figure out how "good" their internet connection is. The idea is simple -- connect to a website and try to transfer as much content as possible within some timeframe. The amount of data transferred can be used to calculate a data transfer speed in \Mbps, where higher is better. Examples include Ookla's aptly-named \url{speedtest.net}, or Netflix's \url{fast.com}.

In theory speed tests are a sound idea, since speed of data transfer is the most immediately obvious thing to a consumer. Connection speed dictates everything from how long it takes pages to load, to how long videos have to buffer, to how fast you can download files. In practice, though, speed testing is challenging and often inaccurate. In order to accurately measure speed, you must have a server receiving and sending test data with at \textit{least} the same speed as the summed connections of all of its users at any one time, posing an immediate performance challenge. Second, the measured connection is only the one between you and whatever data center the speed test server is located in, which may not represent your overall connectivity. Finally, \isps are doubtless familiar with speed test sites and have a serious motive to bias connections in favor of them, to trick their customers into thinking they have better internet than they actually do.

As a result of these combined factors, we decided against using traditional speed tests as any method for assessing internet connectivity.
