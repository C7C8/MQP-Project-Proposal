\section{Internet Architecture}\label{sec:background_internet_architecture}

The internet is formed from a broad range of networks spread across the globe, connected together by very high speed links and data centers traditionally referred to as the "\textbf{\gls{backbone}}." Although the networks are distributed, they do not form a true mesh network so instead data is generally routed through hierarchical networks. These start at the small, local level, and progress upwards (as needed, depending on how far away the destination server is) through increasingly larger networks as data is routed to its final destination, eventually repeating the process in reverse as it approaches the destination server.

Since the internet takes on a hierarchical format, end users are prone to experiencing the bottleneck that is the lowest tier networks they are connected to: their \isp, the network it's connected to, and so on. As a result, internet connections vary in all ways across the United States, with some areas experiencing less than \SI{1}{\mega\bit/\second} down while others may enjoy upwards of \SI{50}{\mega\bit/\second}.