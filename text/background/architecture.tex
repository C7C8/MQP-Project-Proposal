\section{Internet Architecture}\label{sec:background_internet_architecture}

The Internet is comprised of networks spread across the globe, connected together by high speed links and data centers collectively referred to as the "\textbf{backbone}." Although the networks are distributed, they do not form a mesh network. Instead, data is generally routed through hierarchical networks. These networks start at the small, local level, and progress upwards (as needed, depending on the \isp's architecture) through increasingly larger networks as data is routed to its final destination. Data packets eventually repeat the process in reverse as they approach their destination server.

Since the Internet has a hierarchical structure, end users are prone to experiencing a bottleneck that is the networks they are directly connected to: their \isp, its parent network, and so on. As a result Internet connections vary in all ways across the \us, with some areas receiving over 50 \mbps while others receive less than 1 according to our own experiences.