\section{Internet Architecture}\label{sec:background_internet_architecture}

The internet is formed from a broad range of networks spread across the globe, connected together by very high speed links and data centers traditionally referred to as the "\textbf{backbone}." Although the networks are distributed, they do not form a true mesh network. Instead, data is generally routed through hierarchical networks. These start at the small, local level, and progress upwards (as needed, depending on the \isp's architecture) through increasingly larger networks as data is routed to its final destination, eventually repeating the process in reverse as packets approaches their destination server.

Since the internet takes a hierarchical format, end users are prone to experiencing the bottleneck that is the lowest tier networks they are connected to: their \isp, its parent network, and so on. As a result internet connections vary in all ways across the United States, with some areas receiving over 50 \mbps while others receive less than 1. 