\section{Caching}\label{sec:background_caching}

RFC 7234 describes caching as the process of storing "cacheable responses in order to reduce the response time and network bandwidth consumption on future, equivalent requests" \cite{rfc7234}. Caches save frequently-used resources for a given amount of time to reduce the load on upstream resources, from the origin server to Internet infrastructure that provides service along the way. Several components of the Internet may contain caches: the user's browser, \cdns (see \cref{sec:background_cdns}), \dns servers, and others \cite{MozillaFoundation2019HTTPCaching}. These all work together to free up Internet bandwidth and provide a smoother experience to end users.

% Browser caching
Most (if not all) web browsers implement \http caching, saving commonly-used website resources locally \cite{Grigorik2019HTTPCaching}. For example, if users make frequent requests to \texttt{amazon.com}, a cache between Amazon's servers and their users may store a copy of Amazon's logo as it does not change frequently. When \texttt{amazon.com} then requests the logo, the browser's cache responds with the copy, preventing the request from going all of the way to Amazon's servers. This reduces network load that would normally be caused by the request. Of course, sometimes resources do change, so an origin server can give cacheable resources an expiration time after which the cache should validate the freshness resource before fulfilling the request \cite{rfc7234}. 

% CDN caching - not sure if this belongs, maybe just a brief overview before the dedicated CDN section.
Outside of the browser, web resources are often cached by \cdns. These networks are set up by companies that operate major websites and are intended to move the delivery of frequently accessed resources closer to end users, both in network and geographic terms. \cdns are discussed in more detail in \cref{sec:background_cdns}.

% DNS caching
Beyond caching resources that the end user sees and interacts with, other things, like \dns results, use caching as well. As discussed in more detail in \autoref{sec:background_dns}, recursive \dns servers retrieve requests from other \dns servers and often cache them for responding to future requests \cite{rfc1035}. Similar to the expiration time in HTTP caching, \dns caching has a \ttl that forces the \dns resolver to request a fresh answer.