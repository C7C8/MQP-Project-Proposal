\section{Conclusion}\label{sec:comparative-summary}

The takeaway from this analysis has to be that the different methods yielded different results. Each of state \kde charts indicates differences on a state level while the per-method \kde charts in earlier chapters indicate differently shaped distributions. Although the \dns and traceroute methods seem to agree with each other more than either do with the site ping data, they're still different enough to make it impossible to call them the same.

In a way, this makes sense. Consider that all three methods collect fundamentally different kinds of data: traceroutes (as processed here) collect something akin to infrastructure quality, site ping assesses web page loading delays, and \dns collects time for \dns queries to be resolved. \todo{Phrase this in a better way}. At a pure surface level these have select elements in common (such as a bottlenecked connection from a user to their \isp will naturally degrade all three), but beyond that it's only logical that they should report different results.
