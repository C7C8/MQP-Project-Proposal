To measure internet connectivity it is important to concretely define internet connectivity To that end we have put together a list of possible metrics for defining it.

\section{RTT to Everywhere}
\label{sec:definition_rtt_to_everywhere}

\RTT is a measure of how long a packet's round trip between source and destination is. It has a major effect on latency, the quality of streaming applications such as videos, video calls, \voip calls, and multiplayer gaming. Lower \rtts indicate a better connection and a more responsive internet experience.

By measuring a location's average \rtt to any other location, we can roughly determine that connection's average quality. For example, we might find that a particular source has an average \rtt of 50 ms while another has an average 150 ms, so we would say the second has a worse connection than the first.

\paragraph{Normalized by Distance}
Unnormalized \rtt is a valid metric on its own but when averaged across many data points it is difficult to pin a meaningful number on. For instance, if a user lives in San Francisco, CA and has a measured \rtt to Sacramento, CA of 15 ms but an \rtt of 100 ms to Boston, MA (\textapprox3,000 miles away), between the two the user has an average \rtt of 57.5 ms. However, 57.5 ms has no particular meaning to a user because it does not actually show up in the data. That is to say, 57.5 ms does correspond to any real measurement point, and it fails to accurately represent both the Sacramento \rtt and the Boston \rtt.

A possible solution is to normalize \rtts by distance, calculated by dividing \rtt between source and destination by the distance in kilometers between them. This results in a measurement measured in ms/km, which is naturally a small number. This can be thought of as a measurement of infrastructure quality.

%That quality is a better indicator of internet connectivity than raw \rtts are because \rtts will be heavily affected by connections to far away geographic regions. For instance, a ping to a Chinese website will show an \rtt on the order of hundreds of milliseconds. Considering how far away China is and the amount of infrastructure between the \us and China, that's only to be expected. Averages at a location in the \us would be heavily swayed by those large values, meaning raw \rtt would misrepresent the data. The same applies to a user on the west coast trying to connect to a server on the east coast vs. the west. In reality the user's connectivity could be just fine, but by using a simple average or median of the two attempts, the data will be skewed somewhere in the middle.
% 
% To correct for this we can divide \rtt by the distance between source and destination to obtain a measurement of milliseconds per kilometer. Naturally this number will be very small, but it will be a good indicator of infrastructure in a region.

\section{RTT to Regional Locations}\label{sec:definition_rtt_regional}

It may be useful to group \rtt values by region, likely either by distance or national borders, based on the principle that an internet user may be more likely connect to a server in the same region than to one in another region. This measure of raw \rtt shares the same properties as raw \rtt from \cref{sec:definition_rtt_to_everywhere}, except with fewer high-\rtt destinations influencing averages.

\paragraph{Normalized by Distance}
Just like raw global \rtts, regional \rtts can be normalized by distance. Regional \rtt normalized by distance is a good measure of a region's connectivity within itself, could be a better indicator of typical internet connectivity for residents of that region.

\section{Aggregate RTT to /24 Prefixes}\label{sec:definition_rtt_24}

A /24 prefix is the first 3 bytes of a \ipvf address, where the subnet mask (when expanded) is represented as 255.255.255.0. This leaves the last byte of the address free to vary, so a /24 prefix encompasses up to 256 individual addresses.

Aggregate \rtt to /24 prefixes collects data for only one node per /24 prefix, assuming that the other 255 addresses are accurately represented by the single data point from their prefix. This analysis would demonstrate average connectivity between /24 networks, similar in a way to analyzing connectivity between counties or zip codes. In doing so, it would characterize the overall connectivity of that network. Additionally, we could link /24 networks to geographic areas and show which areas have the best and worst /24 prefixes.

\section{RTT to Top Websites}\label{sec:definition_rtt_site_ping}
Five websites (Google, Facebook, Youtube, Yahoo, and Amazon) dominate internet traffic in the \us with over 30\% of the traffic share. 67\% of \us internet users visit Google to search the internet and 68\% use Facebook for social media\cite{Tachalova2017}. These websites (among others) play a major role in internet connectivity and usability for a large portion of internet users in the United States. Therefore, measuring connectivity to these websites provides a window into user experience of the internet for a geographic location.

Beyond direct usage of top websites, major websites adopting \cdns contributes to internet consolidation. According to the Internet Society, 87.5\% of the top 1000 websites in 2018 used \cdns to speed up the delivery of their content. Of these 1000 sites, Amazon Cloudfront and Akamai provide \cdn services to 474 websites. The Internet Society states that four services (Dyn, Akamai, \AWS, and Cloudflare) serve an estimated 50\% of the top 1000 \texttt{.com}, \texttt{.net}, and \texttt{.org} domains \cite{TheInternetSociety2019}.

The reality of a more centralized internet is that, as the web grows ever more consolidated around these cloud services, the ability to speedily connect to any \ip address becomes less of an issue for everyday users. From this vantage point, a better internet connection is one that provides faster load times to the top content providers, regardless of their location. This metric is similar to the \rtt measurements described in \ref{sec:definition_rtt_to_everywhere}, but is more focused on the end user's use case.

\paragraph{Normalized by Distance}
While users might only care about end result and how it impacts their ability to browse the web, normalizing the data by distance to the website data center could provide valuable information regarding regional infrastructure. A region showing high latency to top websites and a high ms/km rating might have room to improve, while a high latency region with low ms/km might not be able to improve. Network architects and companies looking to expand could use this data to prioritize locations, while everyday users could use it to determine if their area has the potential to improve.

\section{Aggregate Regional RTT to Other Regions}

This metric considers average connectivity measurements (such as \rtt) in a political region to all other regions. These regions could be U.S. counties, zip codes, census blocks, or something else. Such aggregation does not necessarily follow the nature of network architecture, but this aggregation would highlight certain potential inequalities on political boundaries. Whatever these inequalities might be, focusing on these regions in this way would be useful to local authorities and constituents looking to improve internet access in their areas.

\section{Advertised speeds per region}
One factor that can often dictate "internet connectivity" is the available "advertised" broadband connection speeds in a given region. It would provide an idea of both the infrastructure in the area and what the \isp can successfully market. Advertised connection speed per region provides a strong metric on what average consumers can access across the US. Comparing the available advertised speeds between regions can show relative infrastructure differences.

\paragraph{Maximum}
The maximum available connection speed across an area gives an idea of what connections are possible. This metric can be used if cost is not considered and all that is desired is the best possible connection.

\paragraph{Minimum}
The minimum connection speed available provides a strong summary of what would be considered to be the most accessible to every American.

\paragraph{Average}
The average advertised connection speed offered by broadband providers in the \us provides a general metric for available speed across the \us. The metric would be the average of all of the connections offered for a region.

% \section{Cost per Megabit per Second}
% Internet infrastructure quality varies widely by location. Generally, urban areas have faster, more reliable, and more stable infrastructure than rural areas. A good metric of the quality of the infrastructure is the cost per \mbps. The same speed connection will be (often significantly) more expensive in an area with a lower quality infrastructure (rural) than in an area with higher quality infrastructure (urban). Comparing the cost per \mbps between different areas reveals the relative infrastructure quality between two different areas. This could also be considered a measure of user experience in the sense that a user must pay more for a better connection, resulting in a detriment to the overall internet usage experience.

% \subsection{Normalized by regional affluence}

%\paragraph{Average}
%The average cost per \mbps in an area can be used as a general metric for the internet connection quality in that area. An average should be taken across connections with "comparable" speeds and prices. For example, an average could include connections between 100 and 1000 \mbps and costs between \$50 and \$200/mo, and leave out outliers such as very expensive or inexpensive connections (relative to the area), or connections much faster or slower than most other connections (e.g. a 2 \mbps \dsl connection in an area where fiber is available). Multiple averages could be taken for a given area. For example, average cost of connections between 10 and 100 \mbps and betwen 100 and 1000. This would allow comparison of "fast" and "slow" connection speeds in two different areas.

% \subsection{Maximum}
% The maximum cost per \mbps in an area shows 

\section{RTT to Internet Backbone}\label{sec:definition_rtt_backbone}

For all non-local traffic -- likely the majority, since major data centers are sparse compared to residential locations -- packets will inevitably pass through the backbone en route to their destination. Backbone \rtt is then a measurement of how long it takes for a user to reach the nearest backbone entry point.

\paragraph{Normalized by distance}

Normalizing \rtt to backbone by distance is as useful as in other \rtt methods, since it gives a better picture of infrastructure quality and avoids outlier biasing. The effect of normalization will likely be diminished, however, since \rtt to backbone is technically a form of regional grouping.

\section{Data Cap by Region}

A data cap on broadband connections could reveal limited bandwidth in an area, and an attempt by the \isp to limit how much traffic consumers are generating. This is not a direct comparison, but possibly a correlation.

\section{IPv6 Availability}\label{sec:definition_ipv6}

Officially established as an internet standard in 2017, \ipvs expands \ipvf to 128 bits in order to address the exhaustion of 32-bit \ipvf addresses. However, deployment of network hardware that supports \ipvs has been slow. According to Google, as of September 2019, the \us has only reached 36.4\% \ipvs adoption \cite{Google2019a}. While this is not an immediate problem, determining where \ipvs has not been implemented in the \us might show which regions are being prioritized. Overall, this definition of connectivity is a form of future internet connectivity.

\section{Connection Stability \& Risk of Disconnection}
Another way of looking at future connectivity involves determining how at risk a region is of becoming disconnected from the internet. A community or region may have sufficient connection now, but if that connection is reliant on a single point of failure (or any degree of failure lower than the rest of the population) in the network, its future connectivity is at risk. Another form of this, although less technical, would involve whether a government entity is capable and willing to sever or curtail internet availability.