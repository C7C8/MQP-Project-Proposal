\section{Design}\label{sec:design_web_ping}
The "site ping" site was developed as a tool for crowd-sourced internet speed testing. The site attempts to measure \rtt between an end user and multiple websites in the top 50 Alexa list. The complete list of the sites used may be found in \cref{sec:sites_list}.

\subsection{RTT Measurement}
There is no standardized protocol for measuring the \rtt to an arbitrary web server using only client-side browser technologies. Our solution was to request objects from web servers and measure the time taken for the client to receive them. Due to \cors restrictions,\footnote{\cors is a security policy implemented by browsers, designed to guard against cross-site scripting attacks by preventing scripts from loading resources or sending data to another domain. For instance, if a malicious user injects a script into another user's view of \texttt{foo.com}, that script would \textit{only} be able to send or receive data from \texttt{foo.com}, and not \texttt{bar.com}.} the only way for client-side Javascript to request external objects is using the \html \texttt{img} tag, which is limited to requesting image resources.

\TCP, the underlying transport protocol used for \http, performs a "handshake" between server and client when establishing a connection. This handshake requires additional packets to be exchanged, adding time to the overall transaction. In addition, modern web sites use \httpse, which adds \tls encryption, and a \tls handshake following the \tcp handshake. As a result, simply measuring the time it takes to request an object includes much more then just the time to transfer the data.

To work around this, the website makes two requests for an object from a web server. The first request includes the \http \texttt{keep-alive} parameter, which keeps the \tcp connection open after the transaction is complete. The second request is timed, and because the connection was kept open the timing results do not include the overhead of the \tcp and \tls handshakes.

Because our goal is to measure \rtt we want the data received by the image tag to be transmitted in a single packet. We wrote a plugin for the Google Chrome browser that uses the Chrome developer tools \api to find the smallest image loaded by each of the websites we wanted to test.

\subsection{Geolocation}
Web ping creates and displays a map of the \us with colored data points representing users. The color of the points corresponds to \rtt, and the point's location is the same as the user's. To determine location we used the geolocation techniques described in \cref{sec:background_geolocation}. The MaxMind database is used in cases where we know the user's \ip address, but not their location. Texas A\&M's \api is used to determine a user's city in cases where the user's device reports a location (e.g. a smartphone) in geographic coordinates. A device-reported location is considered to be more accurate than what is contained in the MaxMind database for mobile devices. \todo{Evan: Make not talk about maxmind live usage}

\subsection{Connection Type Reporting}
Since it is important to compare \rtts for comparable devices (e.g. wired (Ethernet), WiFi, or cellular), web ping makes a best-effort attempt to determine each user's connection type. This is done using the "network information" browser \api. This experimental \api is only enabled by default in Google Chrome and Opera but can be manually enabled on Mozilla Firefox. On mobile devices it is only available on Android browsers, and not iOS.

\subsection{Displaying Results to the User}
The site displays a map to the user. This allows users to see not only real-time results of their test, but compare their results to others in real time. This was intended to create appeal for users.
\todo{Remove this for maxmind reasons?}
