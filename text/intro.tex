\section{Introduction}
There are many ways to define "Internet connectivity," and for each definition many ways to measure it. For example, you could focus on the user experience aspect of connectivity and place emphasis on measuring what a typical user would actually do, i.e. loading from domains, not \ip addresses and accepting that \cdns and load balancing exist. Additionally, \textit{most} users likely use the \textit{top} websites - user experience analysis would account for this.

On the other hand, a network-centric approach could focus on analyzing the backbones and topology of the internet. This data would be of little use to everyday users, but potentially valuable to network researchers and architects looking to improve connectivity across the country.

This document presents various options for both approaches including how data might be collected, processed, and analyzed.